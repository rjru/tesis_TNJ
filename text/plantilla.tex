
%------------ Ejemplo de figura y su referencia.-----
Um modelo de pesquisa similar para LSH é mostrado na Figura \ref{fig:search_model} (b).

\begin{figure}[htp]
\centering
\subfigure[]
    {
    \includegraphics[width=0.32\columnwidth]{images/prueba.png}%\label{fig:slimbe:a}
    \label{fig:am_model}
    }
\subfigure[]
    {
    \includegraphics[width=0.37\columnwidth]{images/prueba.png}%\label{fig:slimbe:a}
    \label{fig:lsh_model}
     }
\caption{Unificação do modelo de consulta por similaridade. (a) Os Métodos de Acesso Métricos (MAMs). (b)  \textit{Locality Sensitive Hashing} (LSH).}
\label{fig:search_model}
\end{figure}

\section{Consultas por Similaridade}\label{sec:consultas-similaridade}

%----------- ejemplo de párrafo.------
\paragraph{Consulta por Abrangência - $Rq(q, r)$}
Consulta que visa recuperar objetos semelhantes a $q$ que se encontram dentro do raio de consulta $r$.  Ou, mais formalmente:

%----------- ejemplo de ecuación. -----------
\begin{equation}
    Rq(q, r) = \{ u \in \mathbb{S} | d(u, q) \leq r \}
\end{equation}

%--------------------- fin de ejemplos...-------

En la Figura \ref{fig:time-series-representation}.

\noindent Assim, com a Equação \ref{eq:sax} pode-se definir palavras de diferentes resoluções, por exemplo para uma série temporal $T$ as palavras SAX com comprimento da palavra $w=4$ e resoluções $2,4,8,16$ são:
\begin{table}[h]
	\begin{tabular}{lclcl}
		$SAX(T, 4, 16)$ & = &$T^{16}$ & =& $\{1100, 1101, 0110, 0001\}$\\
		$SAX(T, 4, 8)$ & = &$T^8$ & = & $\{110, 110, 011, 000\}$\\
		$SAX(T, 4, 4)$ & = &$T^4$ & = & $\{11, 11, 01, 00\}$\\
		$SAX(T, 4, 2)$ & = &$T^2$ & = & $\{1, 1, 0, 0\}$\\
	\end{tabular}
\end{table}


%------------------ ejemplo de tabla...--------

\begin{table}[htbp]
	\centering
	\caption{Trabalhos recentes em indexação de séries temporais.}
	\begin{tabular}{ | l | l | l | c | }
		\hline
		\textbf{  Trabalho  } & \textbf{Dimensão}&\textbf{Dimensão}&\textbf{Método de}\\
		\textbf{ } & \textbf{original}&\textbf{reduzida}&\textbf{redução}\\
		\hline
		iSAX \cite{iSAX} & 480, 960,  & 16, 24,  & iSAX \\
		& 1440, 1920  & 32, 40 & \\
		\hline
		TS-Tree \cite{TSTREE}  &256, 512, 1024  & 16, 24, 32 & PAA \\
		
		\hline
		Scaled and  & 32, 64, 128, & 21, 43, 85, & \textit{Uniform scaling} \\
		Warped Matching \cite{ScalingTimeSeriesQuerying}& 256, 512, 1024 & 171, 341, 683 & \\
		
		\hline
		Exact indexing&  32, 256, 1024 &    16  & PAA \\
		of DTW \cite{KeoghIndDTW} &    &   (todos) & \\
		\hline
	\end{tabular}
	\label{timeseriesIndexing}
\end{table}


%%%%%%%%%%%%%%%%%% ejemplo de algoritmo ---------------------------------
\begin{algorithm}
    \caption{Procurar-1-Motif-Força-Bruta(T, n, R)}
    \begin{algorithmic}[1]
    \REQUIRE  A série temporal $T$, o comprimento do $motif$ $n$ e uma tolerância $R$
    \ENSURE   O \textit{motif}  mais significativo em $T$  o qual tem o maior número de casamentos não triviais .

   \STATE O-melhor-motif-ate-agora = 0;
    \STATE posição-do-melhor-motif-ate-agora = null;

    \FOR{$i = 1$ to comprimento$(T) - n + 1$  }
        \STATE contagem = 0;
        \STATE apontadores  = null;
        \FOR{$j = 1$ to comprimento$(T) - n + 1$  }
            \IF { CasamentoNãoTrivial( $C[i:i+n-1], C[j:j+n+1], R$) }
                \STATE contagem = contagem + 1;
                \STATE apontadores  = anexar(apontadores , j);
            \ENDIF
        \ENDFOR
        \IF {contagem $>$ O-melhor-motif-ate-agora}
            \STATE O-melhor-motif-ate-agora = contagem ;
            \STATE posição-do-melhor-motif-ate-agora = i;
            \STATE motifs = apontadores ;
        \ENDIF
    \ENDFOR
\end{algorithmic}
\label{alg:findmotifs}
\end{algorithm}

%-------------
\Error{Aquí coloco lo que me falta arreglar.}

% --------------- Ejemplo de enumeraciones.----

\begin{enumerate}
	\item  \textbf{SYNT16}  Este conjunto de dados sintético contém 100000 vetores de dimensão 16, distribuídos uniformemente em grupos de 10 em um hipercubo 16-d.
	
	\item  \textbf{SYNT32} Semelhante ao SYNT16, mas contém 100000 vetores de \mbox{dimensão 32.}
	
	\begin{itemize}
		\item AGRODATA-SOUTH. Séries temporais climaticas, fornecidas pelo Agritempo\footnote{\url{http://www.agritempo.gov.br/}}, que consistem de medidas diárias de precipitação e temperatura média coletadas por 19 estações meteorológicas localizadas no sul do estado de São Paulo, no período 1994-2008.
		
		\item AGRODATA-SOUTHEASTERN.  Séries temporais climaticas, também fornecido pelo Agritempo, consitem de medidas diárias de precipitação e temperatura média medida por 10 estações meteorológicas localizadas no sudeste do estado de São Paulo, no período 1995-2009.
	\end{itemize}
	
\end{enumerate}

%--------------------------------- 

\begin{description}
	\item [\itshape A dependência de parâmetros de domínio,] como  \begin{inparaenum}[ a\upshape)]
		\item  o número de funções \textit{hash}; e
		\item  número de subíndices,
	\end{inparaenum}  pode-se  inicialmente  resolver limitando a largura do espaço, mas precisa-se de uma abordagem mais abrangente para a escalabilidade da técnica. Assim, define-se um parâmetro de escala multirresolução para aproximar o grau de adaptabilidade dos parâmetros de domínio conforme a base de dados cresce.   Este parâmetro será obtido usando a dimensionalidade intrínsica dos dados obtida com a teoria de fractais \cite{DBLP:journals/jidm/TrainaTF10}.
	
	
	Os detalhes do processo de configuração do método de indexação, em que se faz a inicialização dos parâmetros de domínio, são apresentados no Algoritmo \ref{alg:initparams}.
	
	
\end{description}

