
\chapter*{Resumen}

\addcontentsline{toc}{section}{Resumen}

En la actualidad la cantidad de datos producida por los medios electr�nicos a tenido un alza importante tanto en n�mero de registros como en complejidad, como consecuencia identificar informaci�n �til de estos grandes conjuntos de datos se ha vuelto un reto, una gran parte de esta aumento en la cantidad de datos viene relacionado a las colecciones de art�culos cient�ficos que recientemente su exploraci�n a despertado el inter�s de la comunidad cient�fica, como consecuencia se han realizado trabajos para facilitar el an�lisis de estas colecciones de documentos usando t�cnicas de miner�a de datos y visualizaci�n de informaci�n, dentro de las t�cnicas de visualizaci�n de datos multidimensionales se encuentran las \textit{Proyecci�n multidimensional} que permiten reducir de una dimensiolidad alta a un espacio de dimensi�n ya sea $1, 2, 3$ conservando las caracter�sticas de similitud de los datos en la dimensi�n original haciendo posible encontrar patrones a trav�s de la capacidad visual humana. La mayor�a de enfoques de proyecciones multidimensional de datos no considera el componente temporal de las colecciones de art�culos cient�ficos a pesar de tener el componente temporal un papel crucial en muchos tipos de datos, en este trabajo se propone incorporar el tratamiento del componente temporal en una proyecci�n multidimensional basado en �rboles filogen�ticos de modo que sea apropiado para tareas de an�lisis exploratoria envolviendo la evoluci�n de temas en colecciones de art�culos cient�ficos.


 
\singlespacing
\vspace*{0.5cm} \noindent \textbf{Palabras Clave:} 
Visualizaci�n temporal de documentos, evoluci�n temporal de temas, �rboles filogen�ticos, proyecci�n de datos multidimensionales.
\chapter*{Abstract}

\addcontentsline{toc}{section}{Abstract}
Currently the amount of data produced by the electronic media had a significant increase in number of records and complexity, as a result identify useful information from these large data sets has become a challenge, a large part of this increase in the amount of data is related to collections of scientific papers recently its exploration aroused the interest of the scientific community, following work has been done to facilitate analysis of these collections of documents using techniques of data mining and information visualization within visualization techniques multidimensional data are the \ textit {multidimensional projection} that reduce a high dimensiolidad to a space dimension either $ 1, 2, 3 $ preserving the characteristics of similarity of the data in the dimension Original making it possible to find patterns through the human visual capacity. Most approaches to multidimensional data projections does not consider the temporal component of the collections of scientific papers despite having the time component a crucial role in many types of data, this paper intends to incorporate the treatment of temporal component in a projection multidimensional based on phylogenetic trees so that it is appropriate for exploratory analysis tasks involving the evolution of issues in collections of scientific articles.

\singlespacing
\vspace*{0.5cm} \noindent \textbf{Keywords:}  Temporal visualization of documents, temporal evolution of topics, phylogenetic trees, multidimensional data projection
